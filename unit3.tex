%\documentclass[hyperref={pdfpagelabels=false}]{beamer}
\documentclass[handout]{beamer}
\let\Tiny=\tiny
\hypersetup{pdfpagemode=FullScreen}
\usepackage[ngerman]{babel}
\usepackage[utf8]{inputenc}
\usepackage{graphics}
\usepackage{listings}
\usepackage{verbatim}
%\setbeamertemplate{navigation symbols}{}

\usetheme{Boadilla}

\usecolortheme{beaver}
\usefonttheme{professionalfonts}
\useinnertheme{rounded}
\useoutertheme{smoothbars}
%\useoutertheme{sidebar}

\definecolor{lGray}{gray}{.90}
\newcommand{\code}[1]{\colorbox{lGray}{\texttt{#1}}}

\author{Christian Kniep}

\begin{document}
\title[Linux Lab Unit 2]{Linux Lab Unit 3 \\ Errorcode, Shellscripting}  
\institute[ICAT Bandung]{Internation Center of Applied Technologies Bandung}
\date[\today]{\today} 

\begin{frame}
	\titlepage
\end{frame} 

\begin{frame}
	\frametitle{Table of content}
	\tableofcontents
\end{frame} 

\section{Shell-Scripting}
    \subsection{automate things}
    \begin{frame}{Create sheudle-Environment}
        \begin{itemize}
            \item<1-> instead of execute the commands you could write them in a textfile and execute them once
            \item<2-> create a file \code{myScript.sh} in your home thats supposed to do
            \begin{enumerate}
                \item<2-> Change to \code{/var/linuxLab/unit3}
                \item<3-> create a directory thats named with your username
                \item<4-> create folders 'monday' to 'tuesday'
            \end{enumerate}
        \end{itemize}
    \end{frame}
\section{Errorcode} 
	\subsection{Introduction}
		\begin{frame}[fragile]
			\frametitle{find}
			\begin{itemize}
				\item<1-> Every command you execute gives back an errorcode from 0-255
                \item<2-> if everything went alright it will be 0
                \item<3-> The other values are free to set
                \item<4-> You can get the EC to the variable 
                    \begin{verbatim}
$ ls -l unit3.tex 
-rw-r--r--  1 kniepbert  staff  1245 10 Aug 21:34 unit3.tex
$ echo $?
0
$ ls -l unitX.tex 
ls: unitX.tex: No such file or directory
$ echo $?
1
\end{verbatim}
            \end{itemize}
		\end{frame}
    \subsection{test}
		\begin{frame}[fragile]
            \frametitle{better way to check}
			\begin{itemize}
                \item<1-> To test in the filesystem there is a better way...
                \item<2-> You wouldnt get output, so its easier to handle
                \item<3-> say hello to \code{test}
                    \begin{verbatim}
$ test -e unit3.tex 
$ echo $?
0
$ test -e unit3.texs 
$ echo $?
1
\end{verbatim}
            \end{itemize}
		\end{frame}
\section{variables}
    \subsection{normal assignment}
        \begin{frame}
            \begin{itemize}
                \item<1-> s
            \end{itemize}
        \end{frame}
    \subsection{assign output}
        \begin{frame}
            \begin{itemize}
                \item<1-> \code{x=`whoami`}
            \end{itemize}
        \end{frame}
\section{comparisions}
    \subsection{dunno}
		\begin{frame}[fragile]
            \frametitle{better way to check}
			\begin{itemize}
                \item<1-> it should look like:
                    \begin{verbatim}
if [ CONDITION ]
    then
        CONSEQUENCE
    else
        ALTERNATIVE
    fi
\end{verbatim}
            \end{itemize}
		\end{frame}

\end{document}