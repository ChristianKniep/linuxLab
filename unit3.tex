\documentclass[hyperref={pdfpagelabels=false}]{beamer}
%\documentclass[handout]{beamer}
\let\Tiny=\tiny
\hypersetup{pdfpagemode=FullScreen}
\usepackage[ngerman]{babel}
\usepackage[utf8]{inputenc}
\usepackage{graphics}
\usepackage{listings}
\usepackage{verbatim}
%\setbeamertemplate{navigation symbols}{}

\usetheme{Boadilla}

\usecolortheme{beaver}
\usefonttheme{professionalfonts}
\useinnertheme{rounded}
\useoutertheme{smoothbars}
%\useoutertheme{sidebar}

\definecolor{lGray}{gray}{.90}
\newcommand{\code}[1]{\colorbox{lGray}{\texttt{#1}}}

\author{Christian Kniep}

\begin{document}
\title[Linux Lab Unit 2]{Linux Lab Unit 3 \\ Errorcode, Shellscripting}  
\institute[ICAT Bandung]{Internation Center of Applied Technologies Bandung}
\date[\today]{\today} 

\begin{frame}
	\titlepage
\end{frame} 

\begin{frame}
	\frametitle{Table of content}
	\tableofcontents
\end{frame} 

\section{Shell-Scripting}
    \subsection{automate things}
    \begin{frame}{Create sheudle-Environment}
        \begin{itemize}
            \item<1-> instead of execute the commands you could write them in a textfile and execute them at once
            \item<2-> create a file \code{myScript.sh} in your home thats supposed to do
            \begin{enumerate}
                \item<2-> Change to \code{/var/linuxLab/unit3}
                \item<3-> create a directory thats named as your user
                \item<4-> create folders 'monday' to 'friday'
            \end{enumerate}
        \end{itemize}
    \end{frame}
\section{Errorcode} 
	\subsection{Introduction}
		\begin{frame}[fragile]
			\frametitle{errorcode}
			\begin{itemize}
				\item<1-> Every command you execute gives back an errorcode from 0-255
                \item<2-> if everything went alright it will be 0
                \item<3-> The other values are free to set
                \item<4-> You can get the EC to the variable 
                    \begin{verbatim}
$ ls -l unit3.tex 
-rw-r--r--  1 kniepbert  staff  1245 10 Aug 21:34 unit3.tex
$ echo $?
0
$ ls -l unitX.tex 
ls: unitX.tex: No such file or directory
$ echo $?
1
\end{verbatim}
            \end{itemize}
		\end{frame}
    \subsection{test}
		\begin{frame}[fragile]
            \frametitle{better way to check}
			\begin{itemize}
                \item<1-> To test in the filesystem there is a better way...
                \item<2-> You wouldnt get output, so its easier to handle
                \item<3-> say hello to \code{test}
                    \begin{verbatim}
$ test -e unit3.tex 
$ echo $?
0
$ test -e unit3.texs 
$ echo $?
1
\end{verbatim}
                \item<4-> For all the different test read \code{man test}
            \end{itemize}
		\end{frame}
\section{variables}
    \subsection{normal assignment}
		\begin{frame}[fragile]
            \frametitle{variable=value}
			\begin{itemize}
                \item<1-> To assign a varbiable with normal values type:
                    \begin{verbatim}
$ var=1
$ echo ${var}
1
$ var = 1
-bash: var: command not found
$ var="Hello World"
$ echo ${var}
Hello World
\end{verbatim}
            \end{itemize}
		\end{frame}

    \subsection{assign output}
		\begin{frame}[fragile]
            \frametitle{var=`cmd`}
			\begin{itemize}
                \item<1-> To assing the \textbf{stdout} use \code{var=`cmd`}
                \item<2-> The stderr will not be assing
                    \begin{verbatim}
$ ls
unit1.tex unit2.tex unit3.tex
$ var=`ls`
$ echo ${var}
unit1.tex unit2.tex unit3.tex
\end{verbatim}
            \end{itemize}
		\end{frame}
\section{conditional programming}
    \subsection{if-then-else}
		\begin{frame}[fragile]
            \frametitle{basic}
			\begin{itemize}
                \item<1-> it should look like:
                    \begin{verbatim}
if [ CONDITION ]
    then
        CONSEQUENCE
    else
        ALTERNATIVE
    fi
\end{verbatim}
            \end{itemize}
		\end{frame}
		\begin{frame}[fragile]
            \frametitle{example}
			\begin{itemize}
                \item<1-> if file exists then echo yes, no instead
                    \begin{verbatim}
$ touch test.txt
$ if [ -e test.txt ]
>    then
>        echo 'yes'
>    else
>        echo 'no'
>    fi
yes
\end{verbatim}
            \end{itemize}
		\end{frame}
		\begin{frame}[fragile]
            \frametitle{example compare variables}
			\begin{itemize}
                \item<1-> some variable-comparisons
                    \begin{verbatim}
$ x=1
$ y=2
$ if [ x == y ]
>    then
>        echo 'yes'
>    else
>        echo 'no'
>    fi
no
\end{verbatim}
            \item<2-> all possible conditions in \code{man test}
            \item<3-> note that the condition 'string-equal' is describted as '=', usualy its '==' which works in bash also.
            \end{itemize}
		\end{frame}
\section{Constructs}
    \subsection{Functions}
        \begin{frame}[fragile]{quick example}
			\begin{itemize}
                \item<1-> lets define a simple function:
                    \begin{verbatim}
$ function func { 
>   echo $1
>   }
$ func "Hello World"
Hello World
$ func Hello World
Hello
\end{verbatim}
            \end{itemize}
		\end{frame}
        \begin{frame}[fragile]{equal}
			\begin{itemize}
                \item<1-> lets define a simple function:
                    \begin{verbatim}
$ function equal {
>   if [ $1 == $2 ];then
>     echo 1
>   else
>     echo 0
>   fi
> }
$ equal 1 1
1
$ equal 1 0
0
$ 
\end{verbatim}
            \end{itemize}
		\end{frame}
    \subsection{for}
        \begin{frame}[fragile]{quick example}
			\begin{itemize}
                \item<1-> for iterates...
                    \begin{verbatim}
$ for item in `ls`;do echo $item;done 
Desktop
Documents
Downloads
Library
Movies
Music
Pictures
Public
Sites
doc\end{verbatim}
            \end{itemize}
		\end{frame}
        \begin{frame}[fragile]{uses output of ls to iterate}
			\begin{itemize}
                \item<1-> if we want to use for to check all dirnames
                    \begin{verbatim}
$ function dirExists {
>   for item in `ls .`;do
>       if [ $1 == $item ];then
>           echo 'found it'
>           fi
>       done
>   }
$ dirExists Music
found it
$ dirExists Musik
$  
\end{verbatim}
            \end{itemize}
		\end{frame}                
        \begin{frame}{back to the schedule-example}
			\begin{itemize}
                \item<1-> Change your \code{MyScript.sh}-Script that it matches teh following goals:
                \begin{enumerate}
                    \item<1-> includes a function that creates and checks the result
                    \item<2-> creates template-files (morning,lunch,afternoon)
                    \item<3-> creates a seperate logfile,where the actions and whether it was successful is stored 
                \end{enumerate}
            \end{itemize}
		\end{frame}
        

\end{document}