\documentclass[11pt]{article}
\usepackage{url}


\usepackage[ngerman]{babel}
\usepackage[utf8]{inputenc}
\usepackage{graphics}
\usepackage{listings}
\usepackage{verbatim}
\usepackage{color}

\author{Christian Kniep}
\title{linuxLab Unit 7}
\date{\today}
%----------------------------
\begin{document}
\maketitle
\section*{linuxLab Test}
If there is now value in braces then you will earn 1 point per default
\subsection*{1. commands}
\begin{enumerate}
    \item Write down the command you will use to... (0.5 each)
    \begin{enumerate}
        \item create a directory:
        \item create a file:
        \item redirect the listing of the current directory to a file:\\
        \item rm a file:
        \item rm a directory:
        \item change permissions (just the basic command)
        \item change the user
    \end{enumerate}
    \item What are the two commands we discussed to search for files or directorys? (2)
    \item What is the difference?
\end{enumerate}
\newpage
\subsection*{2. Manipulate the output}
\begin{enumerate}
    \item The file \texttt{file.txt} contains the linenumerbs 1-10. Use pipes to get only the line 5 and 6 \\
    \texttt{\$ cat file.txt}
    \item There are 100 files in your current directory which contains different numbers. What is the command to find the file(s) which contains 3153? (1.5)\\
    \texttt{\$}
    \item What is the difference between (0.5 each):
    \begin{enumerate}
        \item \texttt{\textgreater}
        \item \texttt{\textgreater\textgreater}
        \item \texttt{1\textgreater}
        \item \texttt{2\textgreater}
    \end{enumerate}
\end{enumerate}
\newpage
\subsection*{3. folders and directorys}
\begin{enumerate}
    \item Write down the lines nesseccary to create the following structure:
    \begin{enumerate}
        \item Making sure you are in your home-directory (0.5)
        \item Creating a directory dir1 and subdirectorys dir1.1 and dir1.2 
        \item Create 30 empty files with a random filename within the two subdirectorys (1.5) \\
        \\
        \\
        \\
        \\
        \\
    \end{enumerate} 
    \item Give a short description of the different blocks of the output: (2) \\
    \texttt{drwxr-xr-x 2 test test 4096 2010-08-03 13:58 fstDir} \\
    \\
    \\
    \\
    \\
    \\
    \item How to change the permissions of fstDir to the following (0.5 each):
    \begin{enumerate}
        \item \texttt{drwxrw-rw-}: \texttt{\$}
        \item \texttt{drwx------}: \texttt{\$}
        \item \texttt{drw-r--r--}: \texttt{\$}
    \end{enumerate}
\newpage
\subsection{4. Environment}
\begin{enumerate}
    \item What is the purpose of the environment? (2) \\
    Hint: Why do you need variables like \$PATH,\$OLDPWD?
    \item How could you manipulate the apearance of your prompt? \\
    \\
    \item What are errorcodes for? \\
    \\
    \\
    \\
    \\
    \\
    \item Assign the errorcode to the variable ec. If ec equals 0 print 'OK' to stdout. (2) \\
    \\
    \\
    \\
    \\
\subsection*{5. Structures}
\begin{enumerate}
    \item Please write down a example of an 'if, then, else'-construct where 0 is compared to 0. (2) \\
    \\
    \\
    \\
    \\
    \\
    \\
    \item and the for-construct? Please add the code for printing every item to stdout... (2) \\
    \texttt{\$ for item in 1 2 3 4;}
\end{enumerate}
\end{enumerate}
    
\end{enumerate}
\end{document}