\documentclass[hyperref={pdfpagelabels=false}]{beamer}
%\documentclass[handout]{beamer}
\let\Tiny=\tiny
\hypersetup{pdfpagemode=FullScreen}
\usepackage[ngerman]{babel}
\usepackage[utf8]{inputenc}
\usepackage{graphics}
\usepackage{listings}
\usepackage{verbatim}
%\setbeamertemplate{navigation symbols}{}

\usetheme{Boadilla}

\usecolortheme{beaver}
\usefonttheme{professionalfonts}
\useinnertheme{rounded}
\useoutertheme{smoothbars}
%\useoutertheme{sidebar}

\definecolor{lGray}{gray}{.90}
\newcommand{\code}[1]{\colorbox{lGray}{\texttt{#1}}}

\author{Christian Kniep}

\begin{document}
\title[Linux Lab Unit 2]{Linux Lab Unit 2 \\ textmanipulation, pipes, output}  
\institute[ICAT Bandung]{Internation Center of Applied Technologies Bandung}
\date[\today]{\today} 

\begin{frame}
	\titlepage
\end{frame} 

\begin{frame}
	\frametitle{Table of content}
	\tableofcontents
\end{frame} 

\section{Filesystem} 
	\subsection{searching}
		\begin{frame}
			\frametitle{updatedb,root}
			\begin{itemize}
				\item<1-> \code{updatedb} create an index of the whole directory-tree
                \item<2-> So you have to be some superuser
                \begin{itemize}
                    \item<2-> The superuse root has no permission-restrictions
                    \item[$\Rightarrow$]<3-> You can't hide from him!
                    \item<4-> Due to security only root can change the owner of a file/directory
                \end{itemize}
            \end{itemize}
        \end{frame}
		\begin{frame}[fragile]
			\frametitle{updatedb,root}
			\begin{itemize}
                \item<1-> if a superuser has performed \code{updatedb} you can locate things:
                    \begin{verbatim}
debian1:/var/linuxLab# locate Network
/var/linuxLab/irene/wednesday/ComputerNetwork
\end{verbatim}
                \item<2-> Please search for \code{bandung} and discuss the problem with \code{updatedb}
                \begin{enumerate}
                    \item<3-> You will find this \code{/var/linuxLab/additional/bandung}
                    \item<4-> But you are not supposed to see it:
                        \begin{verbatim}
stella@debian1:~$ ls /var/linuxLab/additional/bandung
ls: /var/linuxLab/additional/bandung: Permission denied
stella@debian1:~$ ls /var/linuxLab/additional
ls: /var/linuxLab/additional: Permission denied
\end{verbatim}
                \end{enumerate}
            \end{itemize}
		\end{frame}
		\begin{frame}[fragile]
			\frametitle{find}
			\begin{itemize}
                \item<1-> \code{find} searches \texttt{live}
                    \begin{verbatim}
debian1:/var/linuxLab/additional# find ./
./
./bandung
./testDir
./testDir/testFile
\end{verbatim}
                \item<2-> use the manpage / help to search
                \begin{enumerate}
                    \item<2-> find only folders in \code{/var/linuxLab/\textless YOU\textgreater}
                    \item<3-> find only files in \code{/var/linuxLab/\textless YOU\textgreater}
                    \item<4-> search for files which belongs to your user in \code{/var/linuxLab/}
                \end{enumerate}
            \end{itemize}
		\end{frame}
\section{Handle the output}
    \subsection{only last,first}
        \begin{frame}
			\frametitle{tail,head}
			\begin{itemize}
                \item<1-> if you only want to see the last 10 lines of a file: \code{tail file}
                \item<2-> the first 10 lines of a file: \code{head file}
                \item<3-> both provides \code{-n} to provide the amount of lines,e.g. \code{-n50}
                \item<4-> also often used to wait for changes: \code{-F}
                \item<5-> Lets play with it \code{/var/linuxLab/unit2/linenumbers.txt}
                \begin{enumerate}
                    \item<5-> view the last 15 lines
                    \item<6-> view the first 15 lines 
                    \item<7-> view the lines 20-25
                    \item<8-> use \code{-F} to wait for changes
                \end{enumerate}
            \end{itemize}
        \end{frame}
    \subsection{seperate lines}
        \begin{frame}
			\frametitle{grep}
			\begin{itemize}
                \item<1-> to search for patterns within files you use \code{grep \textless pattern\textgreater file}
                \item<2-> try to find the pattern 25 in the previous used file \code{linenumbers.txt}
                \item<3-> try to find your username in the directory \code{/var/log/}, \\
                        for this use \code{grep -R \textless user\textgreater\ \textless path\textgreater *}
            \end{itemize}
        \end{frame}
    \subsection{redirect output}
        \begin{frame}
			\frametitle{output}
			\begin{itemize}
                \item<1-> To redirect the output into a file use \code{\textgreater} \\
                        e.g. \code{ls -l \textgreater\ file}
                \item<2-> what is the difference between \code{\textgreater} and \code{\textgreater \textgreater}?
                \item<3-> actually there are to different outputs:
                \begin{itemize}
                    \item[stdout]<3-> thats the normal output shown by the program
                    \item[stderr]<3-> warnings and errors
                \end{itemize}
                \item<4-> \code{stdout} is refered to with \code{1\textgreater}, \code{stederr} with \code{2\textgreater}
                \item<5-> lets try it:
                \begin{enumerate}
                    \item<5-> \code{ls /var/linuxLab/unit2/*}
                    \item<6-> \code{ls /var/linuxLab/unit2/*\ 1\textgreater\ out }
                    \item<7-> \code{ls /var/linuxLab/unit2/*\ 2\textgreater\ err }
                    \item<8-> \code{ls /var/linuxLab/unit2/*\ 1\textgreater\ out\ 2\textgreater\ err}
                \end{enumerate}
            \end{itemize}
        \end{frame}
    
\end{document}
