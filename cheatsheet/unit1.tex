\documentclass[10pt,landscape]{article}
\usepackage{multicol}
\usepackage{calc}
\usepackage{ifthen}
\usepackage[landscape]{geometry}
\usepackage[ngerman]{babel}
\usepackage[utf8]{inputenc}
\usepackage{graphics}
\usepackage{listings}
\usepackage{verbatim}


% This sets page margins to .5 inch if using letter paper, and to 1cm
% if using A4 paper. (This probably isn't strictly necessary.)
% If using another size paper, use default 1cm margins.
\ifthenelse{\lengthtest { \paperwidth = 11in}}
	{ \geometry{top=.5in,left=.5in,right=.5in,bottom=.5in} }
	{\ifthenelse{ \lengthtest{ \paperwidth = 297mm}}
		{\geometry{top=1cm,left=1cm,right=1cm,bottom=1cm} }
		{\geometry{top=1cm,left=1cm,right=1cm,bottom=1cm} }
	}

% Turn off header and footer
\pagestyle{empty}
 

% Redefine section commands to use less space
\makeatletter
\renewcommand{\section}{\@startsection{section}{1}{0mm}%
                                {-1ex plus -.5ex minus -.2ex}%
                                {0.5ex plus .2ex}%x
                                {\normalfont\large\bfseries}}
\renewcommand{\subsection}{\@startsection{subsection}{2}{0mm}%
                                {-1explus -.5ex minus -.2ex}%
                                {0.5ex plus .2ex}%
                                {\normalfont\normalsize\bfseries}}
\renewcommand{\subsubsection}{\@startsection{subsubsection}{3}{0mm}%
                                {-1ex plus -.5ex minus -.2ex}%
                                {1ex plus .2ex}%
                                {\normalfont\small\bfseries}}
% Special commands
\newcommand{\code}[1]{\texttt{#1}}

\makeatother

% Define BibTeX command
\def\BibTeX{{\rm B\kern-.05em{\sc i\kern-.025em b}\kern-.08em
    T\kern-.1667em\lower.7ex\hbox{E}\kern-.125emX}}

% Don't print section numbers
\setcounter{secnumdepth}{0}


\setlength{\parindent}{0pt}
\setlength{\parskip}{0pt plus 0.5ex}


% -----------------------------------------------------------------------

\begin{document}

\raggedright
\footnotesize
\begin{multicols}{3}


% multicol parameters
% These lengths are set only within the two main columns
%\setlength{\columnseprule}{0.25pt}
\setlength{\premulticols}{1pt}
\setlength{\postmulticols}{1pt}
\setlength{\multicolsep}{1pt}
\setlength{\columnsep}{2pt}

\begin{center}
     \Large{\textbf{ICAT Linux CheatSheet}} \\
\end{center}

\section{Permissions}
\subsection{Ingredience}
\begin{tabular}{@{}llll@{}}
short       & long & file & dir \\ \hline
\verb!r!    & read & read & is shown \\
\verb!w!    & write & write & create stuff in it\\
\verb!x!    & execute & execute it & enter it
\end{tabular}
\subsection{ls-command}
\code{-rw-r--r-- 1 kniepbert staff 10  4 Aug 08:24 unit1.pdf}

\section{Misc}
\subsection{Oneliner}
To concatenate commands in one line use \code{\&\&} and \code{;}
\begin{tabular}{ll}
\verb!cmd1 && cmd2! & fires cmd2 if cmd1 was successfull  \\
\verb!cmd1 ; cmd2! & fires cmd2 whatever result cmd1 has   
\end{tabular}
\subsection{Errorcode}
Every command gives back an Errorcode between 0-255. If everything went alright its 0.
Its stored in the Variable \code{\$?}. \\
\begin{lstlisting}
$ touch test ; echo $?
0
$ ls test ; echo $?
ls: test: No such file or directory
1
\end{lstlisting}
Within your own script you could create an errorcode with \code{exit VALUE}
\subsection{variables}
You could assign values or stdout to variables. Don't use blanks! 
\begin{lstlisting}
$ var=1 ; echo ${var}
1
$ var = 1
-bash: var: command not found
$ var="Hello World" ; echo ${var}
Hello World
$ ls
unit1.tex unit2.tex unit3.tex
$ var=`ls`
$ echo ${var}
unit1.tex unit2.tex unit3.tex
\end{lstlisting}
\section{Compare}
\subsection{if-then-else}
To check a condition and act acording to the result use if-then-else:
\begin{lstlisting}
if [ CONDITION ]
    then
        CONSEQUENCE
    else
        ALTERNATIVE
    fi
\end{lstlisting}
if there is only one command after then/else
\begin{lstlisting}
if [ CONDITION ]
    then CONSEQUENCE
    else ALTERNATIVE
    fi
\end{lstlisting}
for linebreaks use \code{;}
\begin{lstlisting}
if [ CONDITION ]; then cmd1; else cmd2; fi
\end{lstlisting}
\subsection{conditions}
use \code{test} to check and compare. Have a look at the manpage to list all posibile checks.
\begin{lstlisting}
$ if [ -e test ];then echo '1';else echo '0';fi
no
$ touch test
$ if [ -e test ];then echo '1';else echo '0';fi
yes
$ if [ 1 -eq 1 ];then echo '1';else echo '0';fi
yes
$ if [ 2 -eq 1 ];then echo '1';else echo '0';fi
no
$ if [ 'Hi'=='Hi' ];then echo '1';else echo '0';fi
yes
\end{lstlisting}

\end{multicols}
\end{document}
