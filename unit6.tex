\documentclass[draft,hyperref={pdfpagelabels=false}]{beamer}
%\documentclass[handout]{beamer}
\let\Tiny=\tiny
\hypersetup{pdfpagemode=FullScreen}
\usepackage[ngerman]{babel}
\usepackage[utf8]{inputenc}
\usepackage{graphics}
\usepackage{listings}
\usepackage{verbatim}
%\setbeamertemplate{navigation symbols}{}

\usetheme{Boadilla}

\usecolortheme{beaver}
\usefonttheme{professionalfonts}
\useinnertheme{rounded}
\useoutertheme{smoothbars}
%\useoutertheme{sidebar}

\definecolor{lGray}{gray}{.90}

\newcommand{\code}[1]{\colorbox{lGray}{\texttt{#1}}}
\author{Christian Kniep}

\begin{document}
\title{linuxLab Unit 6}
\institute[ICAT Bandung]{Internation Center of Applied Technologies Bandung}
\date[\today]{\today} 

\begin{frame}
	\titlepage
\end{frame} 


\section{Scripting}
    \subsection{some close to easy scripts}
        \begin{frame}{create,find,manipulate and execute}
            Create a folder named as your user in \code{/var/linuxLab/unit6} and...
			\begin{enumerate}
				\item<1-> ...create 10.000 files named \$RANDOM.txt (environment-variable) in 100 folder
                \item<2-> ...find files with odd numbers and make them executable. \\
                            (play around \code{expr 5 \% 2})
                \item<3-> ...find executable files and implement a 'Hello World'-Programm in them
                \item<4-> ...execute every executable file and redirect the output to \code{username.out} in \code{/var/linuxLab/user/}
            \end{enumerate}
		\end{frame}        

\end{document}
