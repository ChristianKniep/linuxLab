\documentclass[hyperref={pdfpagelabels=false}]{beamer}
%\documentclass[draft,handout]{beamer}
\let\Tiny=\tiny
\hypersetup{pdfpagemode=FullScreen}
\usepackage[ngerman]{babel}
\usepackage[utf8]{inputenc}
\usepackage{graphics}
\usepackage{listings}
\usepackage{verbatim}
%\setbeamertemplate{navigation symbols}{}

\usetheme{Boadilla}

\usecolortheme{beaver}
\usefonttheme{professionalfonts}
\useinnertheme{rounded}
\useoutertheme{smoothbars}
%\useoutertheme{sidebar}

\definecolor{lGray}{gray}{.90}
\newcommand{\code}[1]{\colorbox{lGray}{\texttt{#1}}}

\author{Christian Kniep}

\begin{document}
\title[Linux Lab]{Linux Lab}  
\institute[ICAT Bandung]{Internation Center of Applied Technologies Bandung}
\date[\today]{\today} 

\begin{frame}
	\titlepage
\end{frame} 

\begin{frame}
	\frametitle{Table of content}
	\tableofcontents
\end{frame} 


\section{Introduction} 
	\subsection{get to know the terminal}
		\begin{frame}[fragile]
			\frametitle{log in a remote computer}
			\begin{itemize}
				\item<1-> First you have to log on the remote computer \\
                        \begin{verbatim}
$ ssh user@192.168.3.221
user@192.168.3.221's password: 
\end{verbatim}
                \item<2-> Your usernames: \\ \ \\
                \begin{tabular}{|l|l|} \hline
                aegedio & maria \\ \hline
                ryan & sintikhe \\ \hline
                christian & yoseph \\ \hline
                \end{tabular}
            \end{itemize}
		\end{frame}
        \begin{frame}
			\frametitle{look around}
			\begin{itemize}
                \item<1-> who am I \code{whoami}
                \item<2-> where am I \code{pwd} \textbf{P}ath\textbf{W}orking\textbf{D}irectory
				\item<3-> List your home-directory \code{ls}
                \item<4-> OK, its empty. So create a directory \code{mkdir fstDir}
                \item<5-> Whats the difference between
                \begin{itemize}
                    \item<5-> \code{ls}
                    \item<5-> \code{ls -l}
                    \item<5-> \code{ls -la}                    
                \end{itemize}
            \end{itemize}
		\end{frame}
\section{Listings \& Permissions}
    \subsection{ls-Command}
       \begin{frame}[fragile]
			\frametitle{what does it mean?}
			\begin{itemize}
                \item<1-> \code{drwxr-xr-xr-x 2 test test 4096 2010-08-03 13:58 fstDir}
				\item<2-> what are these informations all about?\\
                $\underbrace{\code{d}}_{\text{\tiny{type}}}
                \underbrace{\code{rwx}}_{\text{\tiny{user perm.}}}\ \ 
                \underbrace{\code{r-x}}_{\text{\tiny{group perm.}}}\ \ 
                \underbrace{\code{r-x}}_{\text{\tiny{other perm.}}}\ \
                \underbrace{\code{2}}_{\text{\tiny{\# links}}}\ \ 
                \underbrace{\code{test}}_{\text{\tiny{user}}}\ 
                \underbrace{\code{test}}_{\text{\tiny{group}}}$ \\
                $\underbrace{\code{4096}}_{\text{\tiny{size}}}\ 
                \underbrace{\code{2010-08-03}}_{\text{\tiny{date last altered}}}\ 
                \underbrace{\code{13:58}}_{\text{\tiny{time last altered}}}\ 
                \underbrace{\code{fstDir}}_{\text{\tiny{name}}}$
            \end{itemize}
		\end{frame}
    \subsection{Types}
        \begin{frame}
			\frametitle{a few types you might encounter}
			\begin{itemize}
                \item<1-> Basic types are: \\
                \begin{tabular}{cl}
                    short & description \\ \hline
                    - & regular file \\
                    d & directory \\
                    l & symbolic link (reference)
                \end{tabular}
            \end{itemize}
		\end{frame}
    \subsection{Permissions}
        \begin{frame}
			\frametitle{rwx?}
			\begin{itemize}
                \item<1-> Basic permissions are: \\
                \begin{tabular}{clll}
                    short & long & file-context & dir-context \\ \hline
                    r & read & view the file & the dir is shown in \code{ls} \\
                    w & write & alter the file & create files/dir \\
                    x & execute & execute a file & enter the directory
                \end{tabular}
                \item<2-> Change the permissions \\
                To change the permission use \code{chmod}
                \begin{itemize}
                    \item<3-> User: \code{chmod u=rwx file} or \code{u-r} or \code{u+x}
                    \item<4-> Group / other: \code{g=} / \code{o=}
                \end{itemize}
                \item<5-> Change the group: \code{chgrp \textless group\textgreater file}
            \end{itemize}
		\end{frame}
\section{Doing}
    \subsection{Basic commands}
        \begin{frame}
			\frametitle{cd, move and remove}
			\begin{itemize}
                \item<1-> \textbf{C}hange the \textbf{D}irectory \code{cd destination}
                \item<1-> move (equal to rename) is \code{mv file file1}
                \item<1-> remove (equal to rename) is \code{rm file}
            \end{itemize}
		\end{frame}
        \begin{frame}
			\frametitle{create, read and write}
			\begin{itemize}
                \item<1-> the simplest way to create a file is \code{touch file}
                \item<2-> if you want to edit it you use \code{vim file}
                \begin{itemize}
                    \item<2-> now you are in the neutral mode
                    \item<3-> to edit the file type \code{i}, now there has to be the string \code{-- INSERT --} on the left buttom
                    \item<3-> now insert some stuff into your file
                    \item<4-> if you want to undo the last insertion go to the neutral mode (type \code{ESC} until the left buttom is clean) and type \code{u}
                    \item<5-> to write the file change to the neutral mode and type \code{:w}
                    \item<6-> quitting is the command \code{:q}
                    \item<6-> its possible to do \code{:wq}
                \end{itemize}
            \end{itemize}
		\end{frame}        
    \subsection{practical}
        \begin{frame}
			\frametitle{lets play}
			\begin{itemize}
                \item<1-> change to \code{/var/linuxLab}
                \item<2-> have a look around, what do you see
                \item<3-> try this in the current and in the subdirs
                \begin{enumerate}
                    \item<4-> create a file named as your user 
                    \item<4-> insert your full name to it (use vim)
                    \item<5-> add 1 to the name (rename it to \textless username\textgreater 1)
                    \item<6-> add your name into yout neighbours file
                \end{enumerate}
            \end{itemize}
		\end{frame}
        \begin{frame}
			\frametitle{schedule}
			\begin{itemize}
                \item<1-> lets pin down our leactures during the week
                \begin{enumerate}
                    \item<1-> create a folder as your username \code{/var/linuxLab/\textless user\textgreater}
                    \item<2-> create a folder for each day of the week
                    \item<3-> create a file which contains the leactures and times in the day-folders
                \end{enumerate}
            \end{itemize}
		\end{frame}
    \subsection{additional}
        \begin{frame}
			\frametitle{lazy linux}
			\begin{itemize}
                \item<1-> help! I need somebody!
                \begin{enumerate}
                    \item<2-> The parameter \code{--help} give you a short explaination
                    \item<3-> \code{man \textless command\textgreater} gives you the full information
                \end{enumerate}
                \item<4-> commands \& parameters
                \begin{enumerate}
                    \item<5-> if you type \code{!!} you got the last command 
                    \item<6-> if you type \code{!\$} you got the last parameter (really the last)
                \end{enumerate}
                \item<7-> cd
                \begin{enumerate}
                    \item<8-> type \code{cd} to change to your home
                    \item<9-> type \code{cd -} to change to the previous path you where in
                \end{enumerate}
            \end{itemize}
		\end{frame}
    
\end{document}
