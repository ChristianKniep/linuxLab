%\documentclass[draft,hyperref={pdfpagelabels=false}]{beamer}
\documentclass[handout]{beamer}
\let\Tiny=\tiny
\hypersetup{pdfpagemode=FullScreen}
\usepackage[ngerman]{babel}
\usepackage[utf8]{inputenc}
\usepackage{graphics}
\usepackage{listings}
\usepackage{verbatim}
%\setbeamertemplate{navigation symbols}{}

\usetheme{Boadilla}

\usecolortheme{beaver}
\usefonttheme{professionalfonts}
\useinnertheme{rounded}
\useoutertheme{smoothbars}
%\useoutertheme{sidebar}

\definecolor{lGray}{gray}{.90}

\newcommand{\code}[1]{\colorbox{lGray}{\texttt{#1}}}
\author{Christian Kniep}

\begin{document}
\title{linuxLab Unit 5 \\ practise scripting}  
\institute[ICAT Bandung]{Internation Center of Applied Technologies Bandung}
\date[\today]{\today} 

\begin{frame}
	\titlepage
\end{frame} 


\section{Continue}
    \subsection{writeLines.sh}
        \begin{frame}{what is missing?}
			\begin{itemize}
				\item<1-> create a script \code{writeLines} that reads whole lines from stdin and put them into a file.
                \item<2-> use keyword as first word to control what the script is doing:
                \begin{itemize}
                    \item<2-> maybe use \code{:append:} to append the following line
                    \item<2-> \code{:create:} to create a new file
                    \item<2-> \code{:exit:} to quit
                    \item<2-> ...
                \end{itemize}
            \end{itemize}
		\end{frame}
\section{Kernel and shells}
    \subsection{Kernel}
        \begin{frame}{Bootup}
            \begin{itemize}
                \item<1-> After the boot-up (powerbutton) the BIOS checks some stuff
                \item<2-> The BIOS searches for a little fraction on the HD called MBR
                \item<3-> The MBR contains a referenz to the kernel and an initial filesystem (initrd)
                \item<4-> Since the kernel is started it communicates with the hardware and is kind of the puppetmaster
            \end{itemize}
        \end{frame}
\section{Practise Scripting} 
 	\subsection{some useful commands}
		\begin{frame}[fragile]{split strings}
			\begin{itemize}
                \item<1-> Normaly you have 8 terminals docked onto the kernel (F1-F8)
                \item<2-> The first 6 are commandlines the other usualy graphical
                \item<3-> The login-screen is called getty and the terminals are configured in the \code{/etc/inittab}
                \begin{verbatim}
1:2345:respawn:/sbin/getty 38400 tty1
*snipp*
6:23:respawn:/sbin/getty 38400 tty6
\end{verbatim}
            \end{itemize}
		\end{frame}
    \subsection{Shells}
        \begin{frame}{Why do we need Shells?}
            \begin{itemize}
                \item<1-> Talk to the kernel is like talking to the supernerd
                \item[$\Rightarrow$]<2-> Shells like in \code{/etc/passwd}
                \item<2-> The shell provides
                \begin{enumerate}
                    \item<3-> Command line interpretation
                    \item<4-> Program initiation
                    \item<5-> input-output redirection
                    \item<6-> Pipeline connection
                    \item<7-> Substitution of filenames
                    \item<8-> Maintenance of variables
                    \item<9-> Environment control
                    \item<10-> Shell proraming
                \end{enumerate}
            \end{itemize}
        \end{frame}
    
\end{document}
